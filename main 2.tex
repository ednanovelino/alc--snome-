\documentclass[a4paper,12pt]{article}
\usepackage[top=3.0 cm, bottom=2.0 cm, left=3.0 cm, right=2.0 cm]{geometry}
\usepackage{graphicx,color} %pacote de gráfico
\usepackage{natbib}
\usepackage [brazil] {babel}
\usepackage {enumerate} %numerar todos os itens da sequência
\usepackage {indentfirst}
\usepackage{setspace}
\setstretch{1.5}
\usepackage{amsmath, amsfonts, amssymb}
\usepackage{tensor}
\usepackage [utf8] {inputenc}
\usepackage{mathtools}



\begin{document}

\thispagestyle{empty}
\begin{center}

\Large \textbf {INSTITUTO MILITAR DE ENGENHARIA} \\

\vspace{4 cm}
\Large \textbf {TRABALHO}

\vspace{5 cm}
\Large \textbf{EDNA DOS SANTOS NOVELINO}

\vspace{10 cm}
Rio de Janeiro, RJ \\
2024

\end{center}


\newpage

\thispagestyle{empty}

\begin{center}
\Large \textbf {EDNA DOS SANTOS NOVELINO}

\vspace{3 cm}
\Large \textbf {TRABALHO}
\end{center}

\vspace{3 cm}
\hfill \parbox {8 cm} {Trabalho apresentado ao Instituto Militar de Engenharia } %editar

\vspace{4 cm}


\newpage
\begin{abstract}
    
\end{abstract}


\newpage
\begin{flushleft}
\section{Introdução}
\footnote{Nota 1. }

\subsection{Expressões Matemáticas}

    $$\mathbf{x + y = z^{2}}$$ \\
    $(x + y = z_{2})$ %_ é subescrito

    $$\sum_{n=0}^{\infty} x^{n}$$ % add \displaystule\sum_(i=0)%

\subsection{Ambiente Array e eqnarray}
    $$ \begin{array} {|c|c|} \hline 
    \alpha + B = D & y=x \hline \\ % & próxima coluna% 
    j +p =1 & \beta^{3} +\gama \hline \\
    \end{array} $$
    
    

\newpage
\section{Desenvolvimento}
\begin{enumerate} %P/ especificar os índices, inserir [a/I].
\item item 1
\begin{equation}
    \begin{bmatrix}
        x_{11} & x_{12} & x_{13} & \dots & x_{1n} \\
        x_{21} & x_{22} & x_{23} & \dots & x_{2n} \\
        x_{31} & x_{32} & x_{33} & \dots & x_{3n} \\
    \end{bmatrix}
\end{equation}

\item item 2
\end{enumerate}

\newpage
\section{Resultados}

\newpage
\section{Conclusão}

\end{flushleft} 

\newpage
\section{Referências}


\end{document}
